\chapter{Introduction}\label{Ch:Introduction}

The investor’s ultimate goal is to optimize profit or risk-adjusted return in trading system. Investors construct a portfolio for hands-off or passive investment. A portfolio, a collection of multiple financial assets such as stocks, bonds and bills is usually characterized by its constituents(assets included in a portfolio), weights(the ratio of the total budget invested into each asset), and expected return. Portfolio management is the art and science of making decisions about investment mix and policy, matching investments to objectives, and balancing risk against performance. In this research, we are using reinforcement learning methodologies to optimize portfolios. 

Our research is supported by Magnum Research Limited. It is a fintech company aiming to use advanced AI techniques to help different types of investor to build up a personalised portfolio to optimize their profit in the financial market.

In this project, we are using reinforcement learning to develop an automated trading strategy. The performance of the algorithm is indicated by comparing to the benchmark. For simplifying our situation we just consider the 2-stock portfolio. We also assume that we can buy and sell the stocks at the open and closed price and our behaviour does not affect the stock market. To solve the problem in Reinforcement learning setting the main methodology we used in the report is Q-learning and Deep Q-network. 

Reinforcement learning 
\begin{enumerate}{}
\item a brief description of the sponsoring organization (which often can be derived from the organization's online boilerplate), 
\item a suitably condensed statement of the problem posed by the sponsor, 
\item some discussion of the relevance of the project to the sponsor's business, 
\item the team's approach to the problem, gleaned from the team's Work Statement (including a summary of background study, i.e., literature review, to explain how your work differs from or builds on previous efforts --- this explanation should be reinforced by entries with annotation in the bibliography),
\item summary of the report: in separate paragraphs specify and summarize the major sections of your report (Ch 2, Ch 3, ... , Conclusion,  Appendixes, Glossary, Bibliography).
\end{enumerate}

The team as a whole is responsible for the development of a RIPS report, but various aspects will be divided among individuals, as your team will choose.
A Report Coordinator (RC), selected from the team in the first week of the project, will be responsible for incorporating evolving components of your team's report into the {\LaTeX} template that was used to typeset this document, produced in a particular style we refer to as the {\it RIPS  House Style}.

The template requires familiarity, and so a {\LaTeX} tutorial explaining its use  will be given during the second week of RIPS, and a special section for RCs is available in the ``Templates-etc'' folder on IPAM's R-Drive.
For more information, see  also the  README file.
Prior familiarity with {\LaTeX}  would be advantageous for an RC, but expertise is not necessary because assistance will be provided by RIPS.


\endinput
