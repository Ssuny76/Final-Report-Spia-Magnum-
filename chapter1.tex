\chapter{Introduction}\label{Ch:Introduction}

The investor’s ultimate goal is to optimize profit or risk-adjusted return in trading system. Investors construct a portfolio for hands-off or passive investment. A portfolio, a collection of multiple financial assets such as stocks, bonds and bills is usually characterized by its constituents(assets included in a portfolio), weights(the ratio of the total budget invested into each asset), and expected return. Portfolio management is the art and science of making decisions about investment mix and policy, matching investments to objectives, and balancing risk against performance. In this research, we are using reinforcement learning methodologies to optimize portfolios. 

Our research is supported by Magnum Research Limited. It is a fintech company aiming to use advanced AI techniques to help different types of investor to build up a personalised portfolio to optimize their profit in the financial market.

In this project, we are using reinforcement learning to develop an automated trading strategy. The performance of the algorithm is indicated by comparing to the benchmark. For simplifying our situation we just consider the 2-stock portfolio. We also assume that we can buy and sell the stocks at the open and closed price and our behaviour does not affect the stock market. To solve the problem in Reinforcement learning setting the main methodology we used in the report is Q-learning and Deep Q-network. 

Reinforcement learning is a learning strategy that can let agent to learn without knowing the rule from the environment beforehand but only the reward of actions it make. The objective is to optimize the total reward it gets, and through the process of exploration and exploitation, the agent will learn gradually.

Reinforcement learning has been applied to many situations. The most famous one should be a modified version of well known AlphaGo\footnote{AlphaGo is a copmuter program that able to win the world class Go player. On 23,25,27 May 2017 AlphaGo win the first ranked Go player Ke Jie}, AlphaGo Zero. It involves the technique deep reinforcement learning. Another example is that using deep reinforcement learning, we can train the computer to play atari game\footnote{Google DeepMind is able to do this in 2013}. With wide application of reinforcement learning, someone proposed to apply reinforcement learning in financial sector especially for portfolio management. Some of the previous works attempted to tackle this problem.