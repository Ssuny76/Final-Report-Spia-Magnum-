\chapter{Mathematical Description of the Project}
\label{Ch:mathematical description}
For an automated trading robot with reinforcement learning, investment decisions and actions are made periodically. We allocate fixed amount of budget into two stocks, aiming to maximize return while controlling the volatility and considering the transaction cost. These are the mathematical setting of the portfolio management problem.

\section{Mathematical Formalism}
The portfolio consists of 2 stocks. The closing prices of stocks comprise the price vector for time period $t$, $v_t$. number of stocks.

\section{Transaction Cost}
In a real world, buying or selling stocks is not free. The cost includes commission fee, tax, etc. Assuming a transaction cost proportional to the stock market values exchanged in the market, we set the rate to be 0.2%. We used the preceding study by Angelos to determine the rate. Since we assumed the stock share to be float type, we used formula below to calculate transaction cost.

\section{Benchmark}
The model’s test data performance was compared against two benchmarks. We used the preceding study by Oliver and Hamza to determine the benchmarks. The first, the do-nothing benchmark, allocates half of its starting budget to each stock half-half and then does nothing. This benchmark acted as a very crude approximation of the market since it represents the raw performance of the two stocks.

The second, the rebalance benchmark, re-evaluates its holdings at the end of every market days, and buys or sells stock to ensure the total portfolio value is split into 50-50 between the two stocks. It is important to note that it maintains a proportion of stock values, not stock shares.
